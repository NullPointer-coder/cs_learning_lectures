\documentclass[11pt]{article}
\usepackage[paper=letterpaper, left=1in, right=1in, top=1in, bottom=1in]
           {geometry}
\usepackage[parfill]{parskip}
\usepackage{amsmath}
\usepackage[siunitx]{circuitikz}
\usepackage{color}

\newcommand{\problem}[1]{\textbf{Problem #1 ---} }
\newcommand{\answer}{{\color{red}\textit{Answer: }}}
\newcommand{\amp}{\ampere}

\begin{document}
\thispagestyle{empty}

\begin{center}
{\large CS330 Architecture and Organization}\\
Assignment Chapter 1
\end{center}

\begin{flushright}
Your Name Here %%% <- FIXME
\end{flushright}

\problem{1.1}(2 points) Aside from the smart cell phones used by a billion people, list and describe four other types of computers. (Section 1)

\answer 
\begin{enumerate}
    \item one here
    \item two here
    \item three here
    \item four here
\end{enumerate}

\problem{1.3}(3 points) Describe the steps that transform a program written in a high-level language such as C into a representation that is directly executed by a computer processor. (Section 3)

\answer
Answer here.

\problem{1.4} Assume a color display using 8 bits for each of the primary colors (red, green, blue) per pixel and a frame size of $1280 \times 1024$. (Section 4)

a. (2 points) What is the minimum size in bytes of the frame buffer to store a frame?

\answer
Answer here.

b. (2 points) How long would it take, at a minimum, for the frame to be sent over a 100 Mbits/s network?

\answer
Answer here.

\problem{A} Consider three different processors that implement the same instruction set architecture.  Call the implementations $P_{1}$, $P_{2}$, and $P_{3}$:

\begin{center}
\begin{tabular}{c|r|r}
\textbf{Processor} & \textbf{Clock Rate} & \textbf{CPI} \\ \hline \hline
$P_{1}$ & 3 GHz & 1.5 \\ \hline
$P_{2}$ & 2.5 GHz & 1.0 \\ \hline
$P_{3}$ & 4.0 GHz & 2.2 \\
\end{tabular}
\end{center}
(Section 6)

a. (4 points) Calculate the clock tick time for each processor.  Give your answer in picoseconds, rounded to the closest picosecond.

\answer
Answer here.

b. (2 point) Which processor has the highest performance, if we are interested only in clock cycles per second?

\answer
Answer here.

c. (3 points) Suppose each processor executes a program for 10 seconds.  Calculate the number of clock cycles used. Give your answer rounded to the closest whole number.

\answer
Answer here.

d. (3 points) Calculate the average number of instructions executed per second for each processor.  Which processor has the highest performance, if we are interested only in the average number of instructions executed per second?

\answer
Answer here.

e. (3 points) If the processors each execute a particular program in 10 seconds, find the number of instructions used by each processor.

\answer
Answer here.

\problem{B} Consider two different implementations of an instruction set architecture: $P_{1}$ and $P_{2}$.  The instructions in this ISA can be divided into four different categories:  A, B, C, and D.  The following table gives the clock rate of each processor, along with the CPIs of the instructions from each class.

\begin{center}
\begin{tabular}{r||c|c}
& $P_{1}$ & $P_{2}$ \\ \hline
\textbf{Clock Rate} & 2.5 GHz & 3 GHz \\ \hline
\textbf{CPI for class A instructions} & 1 & 2 \\ \hline
\textbf{CPI for class B instructions} & 2 & 2 \\ \hline
\textbf{CPI for class C instructions} & 3 & 2 \\ \hline
\textbf{CPI for class D instructions} & 4 & 2 \\
\end{tabular}
\end{center}
(Section 6)

When we use the llvm compiler to compile the source code for a particular program, the compiler uses $2.5 \cdot 10^{8}$ instructions, drawn from the four classes as follows:  10\% from A, 20\% from B, 50\% from C, and 20\% from D.

a. (4 points) Calculate the average CPI used during the compilation for both processors?

\answer
Answer here.

b. (4 points) How many clock cycles does the compilation take on each processor?

\answer
Answer here.

c. (4 points) How much time does the compilation take on each processor?

\answer
Answer here.

d. (2 points) Which processor is faster on this task?

\answer
Answer here.

e. (4 points) Calculate the performance of each processor in terms of compilations per second.

\answer
Answer here.

f. (4 points) Use your answer from e.\ to determine which processor has the better performance, and then calculate how much faster that processor is than the slower processor.

\answer
Answer here.

\end{document}
